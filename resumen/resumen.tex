\begin{abstract}

% Contexto 
La alteración de la marcha es el síntoma primario de un enfermo de Parkinson, causando la pérdida de independencia en los sujetos afectados a raíz de caídas, temblores y la rigidez en el movimiento. Durante las sesiones terapéuticas los estímulos externos pueden mejorar las características de la marcha.

% Objetivo 
Este proyecto tiene como objetivo la implementación de PARKIBIP, un sistema de biorretroalimentación, que le permita al enfermo una rehabilitación personal orientada a prolongar el trabajo del terapeuta en su vida cotidiana. 

% Qué se hizo 
Por lo tanto, se realizó una revisión sistemática basada en evidencia científica, que identifica todo el material relevante respecto al uso de dispositivos IMU para analizar la marcha de las personas. En base a la síntesis de estos resultados, se ingenió, diseñó y desarrolló un sistema portable capaz de procesar las mediciones arrojadas por los IMU en tiempo real. Así, detectar las fases de la marcha del paciente y estimularlo a través de sonido y/o vibración, según reglas clínicas pre-definidas por el equipo multidisciplinario de Ingeniería Biomédica. 

% Resultados y conclusiones

Se construyó una herramienta que permite avanzar en el campo de la biomecánica hacia el entendimiento de cómo los estímulos externos favorecen la marcha de las personas con enfermedad de Parkinson. Las pruebas arrojaron excelentes resultados; en promedio se realizaron 20 sesiones, detectando la cantidad exacta de pasos para un 95\% de éstas. Además, PARKIBIP fue evaluado frente a un sistema modelo, obteniendo un error cuadrático medio (RSME) entre las velocidades instantáneas de 0.021.

En conclusión, PARKIBIP es una herramienta eficaz y de gran utilidad para el terapeuta y el paciente. Tanto para retroalimentar la marcha del sujeto afectado por la enfermedad de Parkinson de forma automática, así como también efectuar un análisis objetivo de la marcha del mismo.

\end{abstract}