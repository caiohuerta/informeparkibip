\begin{foreignabstract}

Gait disorders is the primary symptom of Parkinson's patients, causing loss of independence in affected subjects as a result of falls, tremors, and stiffness in movement. During therapeutic sessions external stimuli can improve gait characteristics.

The objective of this project is to implement PARKIBIP, an active biofeedback system, which allows the patient a personal rehabilitation aimed at prolonging the work of the therapist in his daily life.

Therefore, a systematic review based on scientific evidence was carried out, which identifies all relevant material regarding the use of IMU devices to analyze people's gait. Based on the synthesis of these results, a portable system capable of processing the measurements produced by the IMUs in real time was devised, designed and developed. Thus, detect the phases of the patient's gait and stimulate it through sound and vibration, according to clinical rules pre-defined by the multidisciplinary Biomedical Engineering team.

A tool was built that allows progress in the field of biomechanics towards the understanding of how external stimuli favor the gait of people with Parkinson's disease. The tests yielded excellent results; On average, 20 sessions were carried out, detecting the exact number of steps for 95 \% of them. Furthermore, PARKIBIP was evaluated against a model system, obtaining a mean squared error between the instantaneous speeds of 0.021.

In conclusion, PARKIBIP is an effective and useful tool for the therapist and the patient. Both to provide automatically feedback during the gait of the subject affected by Parkinson's disease, as well as to carry out an objective analysis of the patient's progress. 

\end{foreignabstract}