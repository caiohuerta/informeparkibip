\chapter*{Agradecimentos}

El presente trabajo fue realizado bajo el acompañamiento de los profesores Franco Simini y Antonio López Arredondo, pertenecientes al Núcleo de Ingeniería Biomédica de las Facultades de Medicina e Ingeniería, Universidad de la República - URUGUAY. Gracias por su amabilidad, dedicación e ideas.

Un trabajo de investigación es siempre fruto de ideas, proyectos y esfuerzos previos que corresponden a otras personas. En este caso los mas sinceros agradecimientos al D.Sc. Biomédicas Darío Santos y a la Lic. en Fisioterapia Macarena Vergara, que a su vez, formaron parte del equipo interdisciplinario del proyecto PARKIBIP.

Asimismo, se agradece al Prof. Ing. en Computación Sebastian Pizard, de la Facultad de Ingeniería, por la orientación y atención en la realización de la Revisión Sistemática Basada en Evidencia Científica.

El agradecimiento especial a los miembros de la Asociación Uruguaya de Parkinson -AUP-, encabezados por su presidente Ana Maria Martinez, por la colaboración en la investigación. Gracias por la predisposición, amabilidad y empeño en forjar una mejor calidad de vida para con los enfermos de Parkinson.

Finalmente, resta agradecer el apoyo vital que ofrecen las personas que nos estiman, sin el cual no tendríamos la fuerza y energía que nos anima a crecer como personas y como profesionales.

Los más sinceros agradecimientos.