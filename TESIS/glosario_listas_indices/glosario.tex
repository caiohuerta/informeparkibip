\setglossarypreamble[glosario]{Un glosario incluye una lista de términos y su explicación sucinta. El objetivo de este apartado es permitirle al lector especializado en el área, aunque no necesariamente en la temática, comprender con mayor facilidad ciertos términos utilizados en el documento. 
}

\longnewglossaryentry{ANS}
{
type={glosario},
name={Sistema nervioso aut\'onomo (ANS)},
text={\textsc{Sistema nervioso aut\'onomo (ANS)}}
}
{ El sistema nervioso autónomo es la parte del sistema nervioso que controla las acciones involuntarias, tales como los latidos cardíacos y el ensanchamiento o estrechamiento de los vasos sanguíneos.
}

\longnewglossaryentry{AAPB}
{
type={glosario},
name={Association~for~Applied~Psychophysiology~and~Biofeedback},
text={\textsc{Association for Applied Psychophysiology and Biofeedback (AAPB)}}
}
{ 
La Asociación de Psicofisiología Aplicada y Biofeedback (AAPB) se fundó en 1969 como Biofeedback Research Society. Los objetivos de la asociación son promover una comprensión científica del biofeedback y avanzar en los métodos utilizados en la práctica y la aplicación. AAPB es una organización sin fines de lucro.
}

\longnewglossaryentry{biofeedback}
{
type={glosario},
name={Biofeedback},
text={\textsc{biofeedback}}
}
{ El biofeedback es, en resumidas cuentas, una técnica basada en un sistema de sensores gracias a los cuales el paciente es consciente en tiempo real de varios parámetros fisiológicos que describen el funcionamiento de su cuerpo. El individuo es informado en tiempo real acerca de lo que ocurre en varias partes de su cuerpo, aprende a asociar esos fenómenos con ciertas sensaciones y, con un poco de práctica, va siendo más capaz de regular ciertas funciones corporales.
}


\longnewglossaryentry{Wearables}
{
type={glosario},
name={Wearables},
text={\textsc{Wearables}}
}
{ Los Wearables son dispositivos tecnológicos con microprocesadores que, por su tamaño y diseño, se pueden llevar puestos. La palabra proviene del inglés, y hace referencia a accesorios tecnológicos que una persona puede llevar puestos, ``wear'' es el verbo, y ``wearable'' es el adjetivo que en español se traduciría como ponible.
}

\longnewglossaryentry{non_Wearables}
{
type={glosario},
name={Non-Wearables},
text={\textsc{Non-Wearables}}
}
{ Se dice que un dispositivo es Non-Wearable, cuando un dispositivo no es Wearable. Los dispositivos Non-Wearables no cumplen con las condiciones para ser ``ponibles'', ya sea por su tamaño, peso, requerimiento de conexión, etc.
}

\longnewglossaryentry{evidencia}
{
type={glosario},
name={Evidencia},
text={\textsc{Evidencia}}
}
{Una evidencia (del latín, ēvidens, 'visible, evidente, manifiesto') es un conocimiento que se nos aparece intuitivamente de tal manera que podemos afirmar la validez de su contenido, como verdadero, con certeza, sin sombra de duda. En un sentido más restringido se denomina evidencia a cualquier conocimiento o prueba que corrobora la verdad de una proposición.
}

\longnewglossaryentry{SLR}
{
type={glosario},
name={Systematic~literature~review},
text={\textsc{Revisiones sistem\'aticas de la literatura (SLR)}}
}
{La revisión sistemática de la literatura, o revisión sistemática, es un método para identificar, evaluar y resumir el estado del arte de un tema específico en la literatura. En una revisión sistemática, el objetivo es construir una visión general de una pregunta específica y darle un resumen justo de la literatura.
}

\longnewglossaryentry{backward_snowballing}
{
type={glosario},
name={Backward~snowballing},
text={\textsc{backward snowballing}}
}
{Es una estrategia utilizada cómo técnica de búsqueda dentro de una \gls{SLR}. La técnica consiste en partir de un estudio primario relevante encontrado previamente y utilizar sus referencias para encontrar nuevos estudios relevantes.
}

\longnewglossaryentry{forward_snowballing}
{
type={glosario},
name={Forward~snowballing},
text={\textsc{Forward snowballing}}
}
{Es una estrategia utilizada cómo técnica de búsqueda dentro de una \gls{SLR}. La técnica consiste en partir de un estudio primario relevante encontrado previamente y se buscan que estudios lo ubican en sus referencias. 
}

\longnewglossaryentry{SNC}
{
type={glosario},
name={Sistema nervioso central (SNC)},
text={\textsc{Sistema nervioso central (SNC)}}
}
{ En el sistema nervioso central se llevan a cabo los procesos mentales necesarios para comprender la información que se reciben desde el exterior. Asimismo, es el sistema encargado de transmitir ciertos impulsos hacia los nervios y los músculos, por lo que dirige sus movimientos. Se vale de las neuronas (sensoriales y motoras) del encéfalo y la médula espinal para provocar las respuestas precisas a los estímulos que el cuerpo recibe. 
}

\longnewglossaryentry{enfermedad_de_Parkinson}
{
type={glosario},
name={Enfermedad~de~Parkinson},
text={\textsc{Enfermedad de Parkinson}}
}
{La enfermedad de Parkinson (EP) es un trastorno neurodegenerativo multisistémico que afecta al sistema nervioso central (SNC) provocando la aparición de síntomas motores y no motores
}

\longnewglossaryentry{discinesias}
{
type={glosario},
name={Discinesia},
text={\textsc{Discinesia}}
}
{Discinesia, en medicina, es un término utilizado para designar la presencia de movimientos anormales e involuntarios.
}
\longnewglossaryentry{dopamina}
{
type={glosario},
name={Dopamina},
text={\textsc{Dopamina}}
}
{ Las neuronas utilizan un químico cerebral, llamado dopamina, para ayudar a controlar el movimiento muscular. Es un neurotransmisor producido en una amplia variedad de animales, incluidos tanto vertebrados como invertebrados. Cuando se presenta el mal de Parkinson, las neuronas que producen dopamina mueren lentamente. Sin la dopamina, las células que controlan el movimiento no pueden enviar mensajes apropiados a los músculos.
}

\longnewglossaryentry{gait_analysis}
{
type={glosario},
name={Gait~analysis},
text={\textsc{Gait analysis (GA)}}
}
{ El Gait Analysis o Análisis de la Marcha es el estudio sistemático de la locomoción animal, más específicamente el estudio del movimiento humano, utilizando el ojo y el cerebro de los observadores, aumentado por instrumentación para medir los movimientos corporales, la mecánica corporal y la actividad de los músculos. El análisis de la marcha se utiliza para evaluar y tratar a personas con afecciones que afectan su capacidad para caminar. También se usa comúnmente en biomecánica deportiva para ayudar a los atletas a correr de manera más eficiente e identificar problemas relacionados con la postura o el movimiento en personas con lesiones.
}

\longnewglossaryentry{actividades_de_la_vida_diaria}
{
type={glosario},
name={Actividades~de~la~vida~diaria},
text={\textsc{Actividades de la vida diaria (AVD)}}
}
{ Las actividades de la vida diaria, también conocidas como áreas de ocupación, son todas aquellas tareas y rutinas típicas que una persona realiza diariamente y que le permiten vivir de forma autónoma e integrada en la sociedad, cumpliendo así su rol dentro de ella.
}


\longnewglossaryentry{backlog}
{
type={glosario},
name={Backlog},
text={\textsc{Backlog}}
}{La palabra inglesa Backlog significa ``acumulación de algo, especialmente trabajo incompleto o cosas de las que debemos ocuparnos''. Para la Ingeniería de Software, el backlog es una lista de todo el trabajo pendiente, ordenado por prioridad.
}


\longnewglossaryentry{IDE}
{
type={glosario},
name={IDE},
text={\textsc{IDE}}
}{Un entorno de desarrollo integrado o entorno de desarrollo interactivo, en inglés Integrated Development Environment (IDE), es una aplicación informática que proporciona servicios integrales para facilitarle al desarrollador o programador el desarrollo de software. Normalmente, un IDE consiste de un editor de código fuente, herramientas de construcción automáticas y un depurador.
}

\longnewglossaryentry{merge_request}
{
type={glosario},
name={Merge Request},
text={\textsc{Merge Request}}
}
{ En herramientas de versionado de código, un Merge Request es la acción de validar un código que se va a unir de una rama a otra. En este proceso de validación pueden entrar los factores que queramos: Builds (validaciones automáticas), asignación de código a tareas, validaciones manuales por parte del equipo, despliegues, etc.
}

\longnewglossaryentry{aup}
{
type={glosario},
name={Asociaci\'on Uruguaya de Parkinson (AUP)},
text={\textsc{Asociaci\'on Uruguaya de Parkinson (AUP)}}
}
{ La Asociación Uruguaya de Parkinson es una organización uruguaya que tiene como objetivo contribuir a la investigación y la elaboración de material teórico y práctico sobre calidad de vida e intervención en grupos terapéuticos para personas con enfermedad de parkinson  y sus familiares. 
}

\longnewglossaryentry{reeducar_la_marcha}
{
type={glosario},
name={Reeducar~la~marcha},
text={\textsc{Reeducar la marcha}}
}
{
La marcha es seguramente la tarea funcional más corriente. Esta actividad necesita fuerza, movilidad, propriocepción, coordinación y equilibrio. En el momento de la reeducación de la marcha, el fisioterapeuta debe calcular y analizar cada uno de los componentes de la marcha y así define las necesidades propias del paciente. El entrenamiento de la marcha se refiere a la ayuda al paciente para reaprender a caminar con seguridad y eficientemente. 
}

\longnewglossaryentry{AHRS}
{
type={glosario},
name={AHRS},
text={\textsc{AHRS}}
}
{
Los Attitude and Heading Reference Systems (AHRS) o Sistemas de Referencia de Actitud y Rumbo, son sensores tridimensionales que proporcionan información acerca del rumbo, la actitud, y la guiñada de una aeronave. Este tipo de sistemas están específicamente diseñados para reemplazar a los antiguos instrumentos de control giroscópicos, y proporcionar una mejor precisión y fiabilidad.
}

\longnewglossaryentry{ZVD}
{
type={glosario},
name={ZVD},
text={\textsc{Zero Velocity Detection}}
}
{ Zero Velocity Detection o detección de velocidad cero refiere a la detección mediante algoritmos numéricos del estado estacionario de un sensor o dispositivo IMU. 
}

\longnewglossaryentry{kalman-filter}
{
type={glosario},
name={Kalman~filter},
text={\textsc{Kalman Filter}}
}
{En estadística y teoría de control, el filtrado de Kalman, también conocido como estimación cuadrática lineal (LQE), es un algoritmo que utiliza una serie de mediciones observadas a lo largo del tiempo, que contienen ruido estadístico y otras inexactitudes, y produce estimaciones de variables desconocidas que tienden a ser más precisos que los basados en una sola medición, mediante la estimación de una distribución de probabilidad conjunta sobre las variables para cada período de tiempo. El filtro lleva el nombre de Rudolf E. Kálmán, uno de los principales desarrolladores de su teoría.}


\longnewglossaryentry{PoC}
{
type={glosario},
name={PoC},
text={\textsc{PoC}}
}
{
Prueba de concepto (POC), también conocida como prueba de principio, es la realización de un determinado método o idea para demostrar su viabilidad, o una demostración en principio con el objetivo de verificar que algún concepto o teoría tiene potencial práctico. Una prueba de concepto suele ser pequeña y puede estar completa o no.
}

\longnewglossaryentry{background}
{
type={glosario},
name={Background},
text={\textsc{background}}
}
{ Se dice que una instrucción de una aplicación se está ejecutando en background cuando se está ejecutando en un hilo de procesamiento diferente al hilo principal. Este tipo de ejecución permite ejecutar de forma asíncrona o en paralelo las tareas para no sobrecargar al hilo principal.
}

\longnewglossaryentry{wrapper}
{
type={glosario},
name={Wrapper},
text={\textsc{Wrapper}}
}
{En ciencias de la computación, un Wrapper es cualquier entidad que encapsula (envuelve) otro elemento. Los contenedores se utilizan para dos propósitos principales: convertir datos a un formato compatible u ocultar la complejidad de la entidad subyacente mediante la abstracción. Los ejemplos incluyen envoltorios de objetos, envoltorios de funciones y envoltorios de controladores. }


\longnewglossaryentry{product_owner}
{
type={glosario},
name={Product Owner},
text={\textsc{Product Owner}}
}
{ El product owner en el desarrollo de software representa la voz de los clientes y crea una visión del producto junto con los interesados. Cada decisión se toma teniendo en cuenta la visión del producto. }

\longnewglossaryentry{API}{
type={glosario},
name={API},
text={\textsc{API}}
}{API es una sigla que procede de la lengua inglesa y que alude a la expresión Application Programming Interface (cuya traducción es Interfaz de Programación de Aplicaciones). El concepto hace referencia a los procesos, las funciones y los métodos que brinda una determinada biblioteca de programación a modo de capa de abstracción para que sea empleada por otro programa informático.}

\longnewglossaryentry{SI}{
type={glosario},
name={SI},
text={\textsc{Sistema Internacional de Unidades (SI)}}
}{El Sistema Internacional de Unidades (abreviado SI, del francés Système international d'unités) es un sistema constituido por siete unidades básicas: metro, kilogramo, segundo, kelvin, amperio, mol y candela, que definen a las correspondientes magnitudes físicas fundamentales y que han sido elegidas por convención.}

\longnewglossaryentry{.NET}{
type={glosario},
name={.NET},
text={\textsc{.NET}}
}{.NET Framework (pronunciado como "dot net") es un marco de software desarrollado por Microsoft que se ejecuta principalmente en Microsoft Windows.}


\longnewglossaryentry{Gradle}{
type={glosario},
name={Gradle},
text={\textsc{Gradle}}
}{Gradle es un sistema de automatización de desarrollo de software que construye sobre los conceptos de Apache Ant. y Apache Maven.}