\chapter{Introducción}

\section{La enfermedad de Parkinson}

% Descripción de la enfermedad 
La enfermedad de Parkinson (EP) es un trastorno neurodegenerativo multisistémico que afecta al sistema nervioso central (SNC) provocando la aparición de síntomas motores y no motores \cite{E2004}. Causado por la deficiencia de un neurotransmisor, denominado \gls{dopamina} (DA), en el cual la pérdida progresiva de neuronas de dopamina -sito en el mesencéfalo- con la consecuente desaparición dopaminérgica de los ganglios basales, dan lugar a diversos trastornos (i.e. trastornos del movimiento y marcha, fragilidad cognitiva y emocional, alteraciones de la memoria, factores atencionales, entre otros). Es una enfermedad crónica -es decir, que persiste durante un extenso período de tiempo- y progresiva, lo que significa que sus síntomas empeoran con el tiempo \cite{E2004}.  

% Síntomas 
La alteración de la marcha es el síntoma primario del trastorno \cite{Jankovic}, y junto con las caídas causa pérdida de independencia en los sujetos afectados \cite{Ashburn}. Se caracteriza por deficiencias motoras, que incluyen temblor en las extremidades, aumento de la rigidez de las mismas, alteraciones de los parámetros espacio-temporales y fases de la marcha (p. ej. disminución de la longitud del paso y la velocidad de la marcha, variabilidad de zancada a zancada, ralentización del movimiento, marcha arrastrada o un inicio de marcha retrasado) \cite{Rogers,Stamatakis,Hausdorff,Lord}. Asimismo, la interrupción paroxística de la zancada o una marcada reducción en el avance hacia adelante del pie, llamado congelación de la marcha o FOG (del ingles, Freezing of gait) es un evento habitual, que afecta gravemente la calidad de vida, aumenta el riesgo de caídas y fracturas en los sujetos afectados \cite{Perez-Lloret2014,Bloem2004}.

% Algo de Cifras. Población afectada 
Al momento, es la segunda enfermedad neurodegenerativa más frecuente después de la enfermedad de Alzheimer y se estima una predominancia luego de los 60 años \cite{GOMEZGONZALEZ2019396,CHAVEZ-LEON2013}.
En el año 2005, en las cinco naciones más pobladas de Europa occidental -Alemania, Francia, Reino Unido, Italia, y España- y las diez más pobladas del mundo -China, India, Estados Unidos, Indonesia, Brasil, Pakistan, Bangladesh, Rusia, Nigeria y Japón-, que representan 2/3 de la población mundial, el número de personas con EP mayores de 50 años estaba entre 4.1 y 4.6 millones -10 de cada 100 mil habitantes de estos países-. Se espera que este número se duplique para el 2030 \cite{Dorsey2007ProjectedNO}.
A principios del año 2018, un estudio de prevalencia de la EP en Norte América -California, Minesota, Ontario y EE.UU.- para las personas con edad mayor o igual a los 45, estimó que existían alrededor de 680.000 individuos con la EP para el 2010 -equivalente a 477 casos cada 100.000 habitantes mayores de 45-, que ese número aumentaría a unos 930.000 aproximadamente para el 2020 -652 casos cada 100.000- y 1.238.000 para el 2030 -868 sujetos afectados cada 100.000- \cite{Marras2018}. La tendencia creciente, destaca la importancia de optimizar el cuidado y el tratamiento para las personas con EP.
Valga como ejemplo un estudio poblacional realizado en Uruguay (Migues, Canelones) entre los años 1993 y 1995 -única evidencia para Uruguay recuperada-, la prevalencia de la EP alcanzó los 136 casos cada 100.000 habitantes \cite{AljanatiR2012}. 

\section{Evaluación y tratamiento de la enfermedad de Parkinson}
% Problemas en Tratamientos actuales
% Farmacológico, su pérdida de eficacia y efectos secundarios que también son los motores
En la actualidad, el tratamiento clínico de la EP se encuentra segmentado en farmacológico, quirúrgico y rehabilitador. Todas las terapias existentes para el manejo de los síntomas, basadas en el uso de fármacos dopaminérgicos, como la L-Dopa (levodopa) y agonistas de los receptores de dopamina, han demostrado una pérdida de eficacia a lo largo del tiempo. Por cierto, se han asociado con una variedad de efectos secundarios que incluyen \gls{discinesias} y fluctuaciones motoras, que pueden incluso empeorar las deficiencias motoras propias de la enfermedad \cite{Parisi2016}. 

% Fisio y reha
De manera alternativa, para lidiar con la carencia de dopamina, la fisioterapia y la rehabilitación pueden ser efectivas para contrarrestar las deficiencias motoras.

%Reeducar la marcha
En este sentido, la rehabilitación favorece a reinstriur la marcha para lograr mayor autonomía del paciente, prolongar la funcionalidad, reducir la incapacidad motora y desarrollar estrategias para conservar la independencia en las \gls{actividades_de_la_vida_diaria} esenciales. La practica por excelencia, es la aplicación de estímulos externos del tipo auditivos (i.e. metrónomos, golpes rítmicos) o visuales (i.e. líneas perpendiculares, aros marcados en el suelo) con el fin de mejorar la marcha del paciente. Aunque la evidencia científica en relación a la reeducación de la marcha mediante estímulos externos es limitada, ciertos investigadores han reportado que (i) el \gls{SNC} puede verse afectado a través de  la aplicación de estímulos (i.e. táctiles, propioceptivos, auditivos, visuales, entre otros) \cite{Moros2000}, (ii) es posible generar patrones oportunos del ciclo de la marcha \cite{MorrisME1994,Morris1996}, (iii) la propiedad adaptativa del encéfalo para con distintos contextos \cite{Garces2014} y (iv) la recuperación de las características de la marcha y parámetros espacio-temporales mediante el entrenamiento con señales visuales y/o auditivas --de estos pacientes \cite{Ostrosky2000,PALACIOSNAVARRO201649,Mille2012} \cite{Moros2000,Dias2017TREINODM,Almeida2012}.
        
%Evidencia con estímulos
%Evaluación clínica más objetiva y el porqué es importante
%Necesidades
%Dificultades existentes -hoy-
%Uso de escalas subjetivas MDS-UPDRS

%evaluacion clinica        
Los neurólogos solo pueden confiar en observaciones cualitativas en sesiones esporádicas, de corta duración y basadas en su experiencia. Lograr una evaluación clínica con resultados confiables puede ser muy complejo y poco práctico, en virtud de las siguientes limitaciones:
\begin{itemize}
    \item Supervisión continua de los sujetos afectados por parte del personal médico.
    \item Auto-informes realizados por los mismos pacientes -afectado por el sesgo del recuerdo y probablemente sea subjetivo-.
    \item Presencia física de los pacientes a un entorno clínico donde se pueda obtener la evaluación -limitaciones de acceso (i.e condición del paciente, traslados), disponibilidad del médico o laboratorio-.
    \item Evaluación de escalas semi cuantitativa -intrínsecamente subjetivo, con una significativa variabilidad entre los evaluadores en las puntuaciones \cite{Parisi2015}-.
    \item Personal especializado.
    \item Laboratorio de marcha especializado -equipo de alto costo, limitado y compartido entre patologías-.
    \item Mucho tiempo -difícil de lograr en sesiones clínicas-.
\end{itemize}

% Cierre de problemas
La cualidad progresiva de la enfermedad, conlleva a que las características motoras se vuelven cada vez más incapacitantes y las características no motoras menos tratables (por ejemplo, la demencia), acentuando la problemática de la enfermedad. Asimismo, los trastornos de la inestabilidad postural y las alteraciones de la marcha, no se traducen únicamente en alteraciones de aspectos cinemáticos como los mencionados; sino también en aspectos relacionados con su variabilidad: manifestaciones clínicas que de forma progresiva causan limitaciones en las funciones del paciente impactando directamente en su calidad de vida y restringiendo tanto su actividad como su participación social \cite{DILLMANN2014882,MUNOZHELLIN2013190,FernandezDelOlmo2004,GOMEZGONZALEZ2019396}.

De este modo, contar con una evaluación clínica precisa y un seguimiento continuo, es fundamental para identificar una terapia efectiva; conforme a lograr una evaluación cuantitativa más objetiva de la gravedad de los síntomas de la EP, determinar con mayor precisión el grado de progreso de la enfermedad, ajustar la medicación en la atención clínica de rutina y reinstriur la marcha en beneficio de una mejor calidad de vida para los sujetos afectados.
 
\section{Oportunidades tecnológicas de mejorar el tratamiento}
%Ing biomédica y biorretroalimentación

Los sistemas de análisis de movimiento ópticos basados en cámaras son distinguidos como el estándar de oro en la medición del movimiento \cite{Stamatakis}. Si bien son muy adecuados para medir los parámetros espacio-temporales de la marcha en términos de precisión y repetibilidad, requieren sesiones terapéuticas en un entorno de laboratorio, en compañía de personal especializado, equipo considerable y tiempo.

Los avances en los dispositivos de monitoreo \gls{Wearables} están cambiando al sector médico desde un enfoque tradicional reactivo -atención de la salud en el nivel de crisis- a un enfoque de gestión proactiva de la salud \cite{SIMIC,SEPEPE,Galnares,Olivares}; que permite posponer, descubrir y abordar dichos problemas en una etapa temprana \cite{Guo}. La implementación de sistemas basados en esta tecnología innovadora, apuntan a lograr una función terapéutica, tratando a los enfermos a través de la biorretroalimentación (del ingles, \gls{biofeedback}) \cite{Frank2010} \cite{Nonnekes,Ginis2016,Lopez2014,Rochester2010}. Esto es, supervisando las funciones fisiológicas del organismo mediante la utilización de un sistema de retroalimentación, que informa al usuario del estado de la función que se desea controlar.

En el contexto de la salud, principalmente en la EP el biofeedback es fundamental, y se refiere a la provisión de estímulos externos durante o inmediatamente después de un acto motor (p. ej. cuando el sujeto levanta el pie del suelo) que brinden soporte al usuario, estimulen el compromiso cognitivo de los sujetos y potencien los efectos del ejercicio motor \cite{Rochester2010,Nieuwboer2007,Rocha2014,Spaulding2013}. Combinando la naturaleza crónica y neurodegenerativa de la EP, el crecimiento continuo en avances tecnológicos wearables, y que la rehabilitación con ejercicios deben incorporarse a largo plazo en la rutina diaria para alcanzar la máxima eficacia \cite{Tomlinson2012,Lamotte2014}, resulta evidente el GAP clínico y la necesidad de un sistema portable de biofeedback de la marcha en la EP. Cuyo propósito sea aumentar el control voluntario sobre los procesos fisiológicos que de otro modo estarían fuera de la conciencia y/o bajo un control menos voluntario; es decir, reeducar al paciente en rehabilitación a manipular eventos no detectados de forma voluntaria mediante el biofeedback, lograr una función terapéutica superando las limitaciones mencionadas.

La marcha de las personas es un proceso complejo y cíclico que requiere la sinergia de los músculos, los huesos y el sistema nervioso \cite{SAUNDERS1953}, principalmente dirigido a mantener la posición vertical y mantener el equilibrio durante condiciones estáticas y dinámicas \cite{Ayyappa1997}. Asimismo, implica la medición, la descripción y la evaluación de parámetros de marcha (p. ej. velocidad de la marcha, la amplitud de la zancada, la cadencia) que caracterizan la locomoción humana \cite{Ribeiro2017}. La correcta discriminación de las fases de marcha es considerado un punto de partida esencial para distinguir una marcha normal de una patológica o para la evaluación del estado de recuperación de la marcha durante tratamientos de recuperación \cite{Taborri2016}.

\section{Especificación de PARKIBIP}

El presente estudio tiene como objetivo la implementación de PARKIBIP, un sistema que le permita a los pacientes con la enfermedad de Parkinson una rehabilitación personal. Durante las sesiones de fisioterapia los estímulos externos pueden mejorar las características de la marcha y PARKIBIP entonces busca emular estos estímulos en tiempo real para permitirle al enfermo una rehabilitación personal que prolongue el trabajo del fisioterapeuta en su vida cotidiana y mejore la calidad de vida de la persona afectada (i.e. ganar autonomía, estabilidad, independencia, seguridad).

La Unidad de Medición Inercial es un dispositivo que combina varios sensores como el acelerómetro, giroscopio, magnetómetro, entre otros. Estos dispositivos permiten realizar un análisis de la marcha en tiempo real y evaluación de los parámetros espacio-temporales en entornos de la vida real, interiores y exteriores, superando así las limitaciones típicas de las mediciones de laboratorio. Los IMU son más económicos y más prácticos que los sistemas \gls{gait_analysis} completos; solo se requiere una preparación relativamente rápida de los pacientes, ya que el sensor se coloca en el cuerpo por medio de una banda elástica y los datos se pueden transferir fácilmente a través de Bluetooth al software dedicado.

PARKIBIP es el resultado del trabajo de un equipo interdisciplinario a partir de la experiencia de fisioterapeutas que definen las reglas para estimular a los pacientes durante la marcha. Mediante un dispositivo electrónico wearable de bajo costo (Unidad de Medición Inercial), que combina múltiples sensores, se particiona el ciclo de marcha del usuario permitiendo identificar distintos eventos, tales como, el contacto inicial con el suelo o la fase de vuelo. Finalmente, con los eventos detectados, se aplican ciertos algoritmos matemáticos que calculan los parámetros espacio-temporales claves, y en base a un módulo clínico se estimula adecuadamente al usuario. 

\section{Organización del documento}

El documento se organiza en secciones, en donde la Sección \nameref{chap:fundamentos} incluye la revisión de la literatura, los enfoques, las teorías o conceptos pertinentes en que se fundamenta la investigación, así como guiar al lector y ofrecer herramientas analíticas o interpretativas. Dada la complejidad del análisis, se detalla los materiales, procesos y métodos empleados para su resolución en la Sección \nameref{chap:project} y \nameref{chap_RSBE}. Luego, se describe la solución propuesta en el apartado \nameref{chap:implementation}, definiendo la arquitectura y sus componentes, los algoritmos numéricos empleados, el flujo operativo del sistema PARKIBIP, entre otros conceptos relevantes. Finalmente en la Sección \nameref{chap:test_results}, se presentan las pruebas de verificación del sistema, los principales hallazgos, el modo de recolección de datos y el protocolo experimental, así como las sesiones experimentales del proyecto. A modo de epílogo, se sintetizan las posturas expuestas durante el proyecto en el Apartado \nameref{chap:conclusions}, se exponen las complejidades y limitaciones en el transcurso del proyecto. Asimismo, se efectúan contribuciones teóricas o metodológicas a la disciplina y recomendaciones para profundizar en el campo del biofeedback aplicado a la EP.
