\chapter{Pruebas funcionales: Matriz de casos de prueba}\label{apendice:tests-cases}

De acuerdo con la etapa de planeación, fueron diseñados los casos de prueba en la herramienta tabular Excel, mediante una matriz de casos de prueba con la definición de los atributos: identificador, escenario, caso de test, pre-condición, prioridad, etapa, resultado esperado, entre otros. La tabla Tab.\ref{tab:design-cases} muestra ciertas pruebas funcionales que se ejecutaron en PARKIBIP.

\fontsize{9}{11}\selectfont
\setlength\LTleft{-2.5cm}
\begin{longtable}{| p{2cm} | p{2cm}| p{2.5cm}| p{4.5cm}| p{3cm} | p{1.5cm}|}
\caption{Casos de prueba funcionales del sistema PARKIBIP, definidos por los campos identificador del caso, modulo del sistema, escenario de prueba, caso dentro del escenario, etapa, pre-condición	y prioridad.
}\\\hline
\label{tab:design-cases}
\textbf{Id\_caso\_test} & \textbf{Módulo}& \textbf{Escenario} &  \textbf{CASO} & \textbf{Pre-condición} & \textbf{Prioridad} \\ \hline
\endhead
TC\_AUTH\_001 & Autenticación  & Verificar login con Auth0 & En la pantalla de inicio a la aplicación, se ingresa un usuario y contraseña validos & Usuario y contraseña validos en PARKIBIP & BAJA \\ \hline
TC\_AUTH\_002 & Autenticación  & Verificar login con Auth0 & En la pantalla de inicio a la aplicación, se ingresa un usuario y contraseña inválidos & Usuario y contraseña inválidos para PARKIBIP & BAJA \\ \hline
TC\_AUTH\_003 & Autenticación  & Verificar logout con Auth0 & Dentro de una sesión de aplicación y en el menú de navegación, se ejecuta el botón cerrar sesión & Sesión vigente en el sistema & BAJA \\ \hline
TC\_AUTH\_004 & Autenticación  & Chequear descifrado de token JWT & En caso de un inicio de sesión satisfactorio, el proveedor de \textit{Idaas} -Auth0- en el paso final del flujo de autenticación un token del tipo JWT. PARKIBIP interpreta el mismo, para adquirir información del usuario. Se desea chequear dicha interpretación.  & Usuario y contraseña validos en PARKIBIP. Proceso de validación de ingreso en curso & BAJA \\ \hline
TC\_AUTH\_005 & Autenticación  & Distinguir niveles de Usuarios con Auth & Previo a finalizar el flujo de autenticación y autorización entre PARKIBIP-Auth0, se desea verificar los roles del usuario, para una posterior distinción dentro de la aplicación. Roles: Terapeuta/Paciente & Usuario y contraseña validos en PARKIBIP. Proceso de validación de ingreso en curso & BAJA \\ \hline
TC\_IMU\_001 & IMU & Búsqueda bluetooth de dispositivos IMU & En la pantalla inicial, se oprime el botón cuyo icono es un pie & Sin una conexión establecida. Bluetooth activado & ALTA \\ \hline
TC\_IMU\_002 & IMU & Búsqueda bluetooth de dispositivos IMU & En la pantalla inicial, se oprime el botón cuyo icono es un pie. Se repite el proceso con el otro pie & Sin una conexión establecida. Bluetooth activado & ALTA \\ \hline
TC\_IMU\_003 & IMU & Establecer conexión a dispositivo IMU & Como resultado de una búsqueda de dispositivos, el sistema lista los hallados. Se pretende probar la conexión a un único dispositivo inercial & Sin una conexión establecida. Bluetooth activado & ALTA \\ \hline
TC\_IMU\_004 & IMU & Establecer conexión a dispositivo IMU & Como resultado de una búsqueda de dispositivos, el sistema lista los hallados. Se pretende probar la conexión a múltiples dispositivos inerciales & Sin una conexión establecida. Bluetooth activado & ALTA \\ \hline
TC\_IMU\_006 & IMU & Re-conexión a un dispositivo IMU & Luego de una interrupción involuntaria e inesperada, se desea recuperar la conexión con el dispositivo.  & Conexión establecida & MEDIA \\ \hline
TC\_IMU\_007 & IMU & Verificar Productores  de datos inerciales & Se desea probar la recuperación de datos de aceleración del sensor acelerómetro & Conexión establecida.
Sensor acelerómetro configurado en la placa del IMU, con cierta frecuencia de salida de datos & ALTA \\ \hline
TC\_IMU\_008 & IMU & Verificar Productores  de datos inerciales & Se desea probar la recuperación de datos de velocidad angular del sensor giroscopio & Conexión establecida.
Sensor giroscopio configurado en la placa del IMU, con cierta frecuencia de salida de datos y rango de valores establecido & ALTA \\ \hline
TC\_IMU\_009 & IMU & Verificar Productores  de datos inerciales & Se desea probar la recuperación de datos de fuerzas magnéticas  del sensor magnetómetro & Conexión establecida.
Sensor magnetómetro configurado en la placa del IMU, con cierta frecuencia de salida de datos & ALTA \\ \hline
TC\_IMU\_010 & IMU & Verificar Productores  de datos inerciales & Luego de recopilar información fragmentada, se desea integrar la recuperación de datos desde múltiples sensores inerciales & Conexión establecida.
Sensor acelerómetro, magnetómetro, giroscopio configurados en la placa del IMU. & ALTA \\ \hline
TC\_IMU\_011 & IMU & Motor de vibración & Vibración local al IMU & Conexión establecida.
Motor de vibración establecido, mediante la potencia y tiempo de vibración. & ALTA \\ \hline
TC\_QUAT\_001 & Quaternion & Chequear calibración inicial de quaternion & Estimación inicial de la Orientación en base a las primeras medidas. Se desea probar la correctitud numérica  & Un conjunto de mediciones & ALTA \\ \hline
TC\_QUAT\_002 & Quaternion & Validar conversión de representaciones & Conversión de un quaternion estimado a matriz de rotación.  Se desea probar la correctitud numérica  & Quaternion estimado & ALTA \\ \hline
TC\_QUAT\_003 & Quaternion & Validar conversión de representaciones & Obtención del quaternion inverso  o conjugado a el mismo. Se desea probar la correctitud numérica  & Quaternion estimado & ALTA \\ \hline
TC\_QUAT\_004 & Quaternion & Validar conversión de representaciones & Transformación de quaternion a sus respectivos ángulos de euler. Se desea probar la correctitud numérica  & Quaternion estimado & ALTA \\ \hline
TC\_MEASURE\_001 & Medición & Validar conversión de grados a radianes & Conversión de las unidades grados a radianes. Útil para transformar las velocidades angulares según el proveedor del IMU.  Se desea probar la correctitud numérica  & Medición velocidad angular en grados & ALTA \\ \hline
TC\_MEASURE\_002 & Medición & Verificar conversión a SI de unidades & Conversión de las unidades de las observaciones, a unidades estándar de PARKIBIP.  Se desea probar la correctitud numérica  & Mediciones aceleración, velocidad angular, fuerzas magnéticas tridimensionales. & ALTA \\ \hline
TC\_CONFIG\_001 & Configuration & Verificar modificación de parámetros & Dentro de la pantalla de Configuración, se pretende editar los valores de los parámetros, almacenarlos  y luego recuperar el estado modificado.  &  & MEDIA \\ \hline
TC\_CONFIG\_002 & Configuration & Verificar selección de algoritmo numérico & Dentro de la pantalla de Configuracion, se pretende seleccionar dentro de un listado un método particular y luego que el sistema lo recupere para futuros procesamientos. &  & MEDIA \\ \hline
TC\_CONFIG\_003 & Configuration & Verificar recuperación de estado previo & Dentro de la pantalla de Configuración, se pretende editar los valores de los parámetros sin almacenarlos,  luego deshacer los cambios y recuperar el estado previo.  & Edición de parámetros en curso & BAJA \\ \hline
TC\_CONFIG\_004 & Configuration & Verificar recuperación de estado de fabrica & Dentro de la pantalla de Configuración, se pretende restablecer los valores por defecto.  & Modificación previa de parámetros & BAJA \\ \hline
TC\_BIOFEEDBACK\_001 & Biofeedback & Verificar ejecución de reglas de detección & Se desea validar la detección de fases de la marcha, en base a condiciones lógicas sobre los datos resultantes de los diversos algoritmos. & Conjunto de datos de ZVD. Conjunto de estados de Kalman. Datos inerciales. & ALTA \\ \hline
TC\_BIOFEEDBACK\_002 & Biofeedback & Verificar correctitud de parámetros espacio-temporales & Calculo de velocidad instantánea y media.  Se desea probar la correctitud numérica  & Conjunto de estados de Kalman & ALTA \\ \hline
TC\_BIOFEEDBACK\_003 & Biofeedback & Verificar correctitud de parámetros espacio-temporales & Calculo de  cadencia.  Se desea probar la correctitud numérica  & Numero de pasos. Duración. & ALTA \\ \hline
TC\_BIOFEEDBACK\_004 & Biofeedback & Verificar correctitud de parámetros espacio-temporales & Calculo de aceleración media.  Se desea probar la correctitud numérica  & Mediciones de aceleración & ALTA \\ \hline
TC\_BIOFEEDBACK\_005 & Biofeedback & Verificar correctitud de parámetros espacio-temporales & Calculo de numero de pasos.  Se desea probar la correctitud numérica  & Conjunto de valores de ZVD & ALTA \\ \hline
TC\_BIOFEEDBACK\_006 & Biofeedback & Verificar correctitud de parámetros espacio-temporales & Distancia recorrida. Se desea probar la correctitud numérica  & Conjunto de estados de Kalman. Duración & ALTA \\ \hline
TC\_FUSER\_001 & DataFuser & Verificar el fusionado con fuentes de datos  & A partir de 2 productores de datos configurados  -acc,gyro-.  se desea construir un único paquete de observaciones en un instante dado.  & Vectores tridimensionales aceleración y velocidad angular, además del timestamp correspondiente & ALTA \\ \hline
TC\_FUSER\_002 & DataFuser & Verificar el fusionado con fuentes de datos  & A partir de los productores de datos configurados -acelerómetro, giroscopio, magnetómetro, se desea construir un único paquete de observaciones en un instante dado. Se desea probar la correctitud numérica  & Mediciones aceleración, velocidad angular, fuerzas magnéticas tridimensionales, además del timestamp correspondiente & ALTA \\ \hline
TC\_ZVD\_001 & ZVD & Chequear  despachos de eventos de ZVD & Probar el funcionamiento de la cola circular de despacho de eventos de ZVD & Recepción iterativa de resultados de ZVD & ALTA \\ \hline
TC\_ZVD\_002 & ZVD & Verificar Resultado del test estadístico & Chequeo de la correctitud numérica del algoritmo numérico MV & Conjunto de datos de aceleración. Configuraciones requeridas & ALTA \\ \hline
TC\_ZVD\_003 & ZVD & Verificar Resultado del test estadístico & Chequeo de la correctitud numérica del algoritmo numérico GLRT & Conjunto de datos de aceleración y giroscopio Configuraciones requeridas & ALTA \\ \hline
TC\_ZVD\_004 & ZVD & Validar valores estacionarios con datos Open Source de MV & Se desea probar el algoritmo numérico MV mediante datos de prueba reales y de acceso libre & Conjunto de datos inerciales Open Source & ALTA \\ \hline
TC\_ZVD\_005 & ZVD & Validar valores estacionarios con datos Open Source de GLRT & Se desea probar el algoritmo numérico GLRT mediante datos de prueba reales y de acceso libre & Conjunto de datos inerciales Open Source & ALTA \\ \hline
TC\_ZVD\_006 & ZVD & Validar valores estacionarios con Metawear de MV & Se desea probar el algoritmo numérico MV mediante datos recibidos por el IMU de PARKIBIP & Caso de test 10 y 15 & ALTA \\ \hline
TC\_ZVD\_007 & ZVD & Validar valores estacionarios con Metawear de GLRT & Se desea probar el algoritmo numérico GLRT mediante datos recibidos por el IMU de PARKIBIP & Caso de test 10 y 15 & ALTA \\ \hline
TC\_HANDLER\_001 & HandlerThread & Verificar envío de datos desde DataAnalyzer a KalmanThread & Verificación de el envío de datos de medición entre hilos distintos de ejecución. Desde un DataAnalyzer a un KalmanFilter & Medición en un DataAnalyzer & ALTA \\ \hline
TC\_HANDLER\_002 & HandlerThread & Verificar envío de datos desde KalmanThread a DataAnalyzer & Verificación de el envío de datos de medición entre hilos distintos de ejecución. Desde un KalmanFilter a un DataAnalyzer & Estado de Kalman en KalmanThread & ALTA \\ \hline
TC\_ORIENTATION\_001 & Orientation Filter & Verificar el computo de la Orientación del sensor & Se desea probar el filtro de orientación mediante datos recibidos por el IMU de PARKIBIP & Rotación gráfica en tiempo real & ALTA \\ \hline
TC\_ACCELERATION\_001 & Acceleration & Verificar la extracción de la aceleración del usuario & Se desea probar el método de rotación al marco inercial y remoción de componentes gravitatorias. & Caso de test 12 & ALTA \\ \hline
TC\_KALMAN\_001 & KalmanFilter & Verificar Calibración inicial de Filtro de Orientación para Kalman & Se desea probar la integración del algoritmo de calibración al filtro de Kalman & Caso de test 17 & ALTA \\ \hline
TC\_KALMAN\_002 & KalmanFilter & Verificar detecciones de cero velocidad (ZVD)  para Kalman & Se desea probar la integración del algoritmo de cero velocidad MV al filtro de Kalman & Caso de test 40 & ALTA \\ \hline
TC\_KALMAN\_003 & KalmanFilter & Verificar detecciones de cero velocidad (ZVD)  para Kalman & Se desea probar la integración del algoritmo de cero velocidad GLRT al filtro de Kalman & Caso de test 41 & ALTA \\ \hline
TC\_KALMAN\_004 & KalmanFilter & Verificar la actualización del Filtro de Orientación para Kalman & Se desea probar la integración del algoritmo filtro de orientación & Caso de test 44 & ALTA \\ \hline
TC\_KALMAN\_005 & KalmanFilter & Verificar la obtención de la aceleración del sensor para Kalman & Se desea probar la integración del método de obtención de la aceleración del usuario & Caso de test 45 & ALTA \\ \hline
TC\_KALMAN\_006 & KalmanFilter & Verificar Matriz de Covarianza P y Q & Chequeo de la correctitud numérica del calculo de las matrices de covarianza & Caso de test 15 & ALTA \\ \hline
TC\_KALMAN\_007 & KalmanFilter & Verificar Innovation Z & Chequeo de la correctitud numérica del calculo del vector de invocación.  & Caso de test 15 & ALTA \\ \hline
TC\_KALMAN\_008 & KalmanFilter & Verificar algoritmo Predictor -ecuaciones mecánicas- & Chequeo de la correctitud numérica del computo de las ecuaciones mecánicas.  & Caso de test 15 & ALTA \\ \hline
TC\_KALMAN\_009 & KalmanFilter & Verificar algoritmo Corrector sin Compensación de velocidad & Chequeo de la correctitud numérica del método Corrector del filtro de Kalman & Caso de test 15 & ALTA \\ \hline
TC\_KALMAN\_010 & KalmanFilter & Verificar Compensación de velocidad en Kalman Filter & Chequeo de la correctitud numérica del método Corrector del filtro de Kalman, aplicando la compensación de ZVD & Caso de test 15. Caso 40/41 & ALTA \\ \hline
TC\_KALMAN\_011 & KalmanFilter & Verificar estados de Kalman Filter con datos Open Source de marcha & Se desea validar el comportamiento de la rotación inercial bajo una  gravedad negativa & Conjunto de datos inerciales Open Source & ALTA \\ \hline
TC\_KALMAN\_012 & KalmanFilter & Verificar estados de Kalman Filter con datos Open Source de marcha & Verificación de la correctitud numérica del método que calcula la velocidad promedio & Conjunto de datos inerciales Open Source & ALTA \\ \hline
TC\_KALMAN\_013 & KalmanFilter & Verificar estados de Kalman Filter con datos Open Source de marcha & Verificación de la correctitud numérica del método que calcula la trayectoria & Conjunto de datos inerciales Open Source & ALTA \\ \hline
TC\_KALMAN\_015 & KalmanFilter & Verificar estados de Kalman Filter con datos Open Source de marcha & Verificación de la correctitud numérica del método que calcula la velocidad instantánea & Conjunto de datos inerciales OpenSource & ALTA \\ \hline
TC\_KALMAN\_016 & KalmanFilter & Verificar estados de Kalman Filter con datos AUP y metawear & Pruebas experimentales con el factor Beta = 0.1 recomendado por el creador del filtro de orientación & Conjunto de mediciones inerciales & ALTA \\ \hline
TC\_KALMAN\_017 & KalmanFilter & Verificar estados de Kalman Filter con datos AUP y metawear & Pruebas experimentales con el factor Beta = {1,3,5,6} &  & ALTA \\ \hline
TC\_KALMAN\_018 & KalmanFilter & Verificar estados de Kalman Filter con datos AUP y metawear & Se desea validar el comportamiento de la rotación inercial bajo una  gravedad negativa para el sensor de PARKIBIP & Quaternion estimado. Medición inercial aceleración & ALTA \\ \hline
TC\_KALMAN\_019 & KalmanFilter & Verificar estados de Kalman Filter con datos AUP y metawear & Se desea validar el comportamiento de la rotación inercial bajo una  gravedad positiva para el sensor de PARKIBIP & Quaternion estimado. Medición inercial aceleración & ALTA \\ \hline
TC\_KALMAN\_020 & KalmanFilter & Verificar estados de Kalman Filter con datos AUP y metawear & Verificación de la correctitud numérica del computo de los vectores de estados (posición, velocidad, orientación), resultantes del filtro de Kalman & Vector de estado Kalman,  ZVD y mediciones asociadas. & ALTA \\ \hline
TC\_KALMAN\_021 & KalmanFilter & Verificar estados de Kalman Filter con datos AUP y metawear & Verificación de la correctitud numérica del método que aproxima la trayectoria en base a la posición y duración de la sesión de rehabilitación, empleando el sensor IMU de PARKIBIP. & Vector de estado Kalman y datos de sesión activa & ALTA \\ \hline
TC\_KALMAN\_023 & KalmanFilter & Verificar estados de Kalman Filter con datos 2 dispositivos & Integración de Kalman Filter para aproximar la trayectoria en base a la posición y duración de la sesión de rehabilitación, combinando ambos dispositivos IMU de PARKIBIP. & Vector de estado Kalman y datos de sesión activa & ALTA \\ \hline
TC\_KALMAN\_024 & KalmanFilter & Verificar estados de Kalman Filter con datos 2 dispositivos & Integración de Kalman Filter para aproximar la velocidad promedio en base a la instantánea de la sesión de rehabilitación, combinando ambos dispositivos IMU de PARKIBIP. & Vector de estado Kalman y datos de sesión activa & ALTA \\ \hline
TC\_KALMAN\_025 & KalmanFilter & Verificar estados de Kalman Filter con datos 2 dispositivos & Integración de Kalman Filter para aproximar la velocidad promedio en base a la instantánea de la sesión de rehabilitación, empleando ambos dispositivos IMU de PARKIBIP. & Vector de estado Kalman y datos de sesión activa & ALTA \\ \hline
TC\_POOL\_001 & PoolStorage & Chequear almacenamiento de datos & Se desea verificar el almacenamiento adecuado de mediciones bajo el soporte de un servicio recursivo de fondo. Su procesamiento es en batch & Medición inercial. Servicio de fondo iniciado & ALTA \\ \hline
TC\_POOL\_002 & PoolStorage & Chequear almacenamiento de datos & Se desea verificar el almacenamiento adecuado de cero velocidades, bajo el soporte de un servicio recursivo de fondo. Su procesamiento es en batch & Medición inercial. Servicio de fondo iniciado & ALTA \\ \hline
TC\_POOL\_003 & PoolStorage & Chequear almacenamiento de datos & Se desea verificar el almacenamiento adecuado de los vectores de estado de Kalman, bajo el soporte de un servicio recursivo de fondo. Su procesamiento es en batch & Medición inercial. Servicio de fondo iniciado & ALTA \\ \hline

\end{longtable}
\normalsize