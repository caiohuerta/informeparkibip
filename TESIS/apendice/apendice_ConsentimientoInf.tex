\chapter{Documento - Consentimiento informado al usuario de PARKIBIP}\label{anexo:Consentimiento}

PARKIBIP: Retroalimentación activa en la marcha de personas con Enfermedad de Parkinson.  

Proyecto de investigación llevado a cabo por Samuel Sainz, Carlos Huerta, Lic. en fisioterapia Macarena Vergara, Lic. en fisioterapia  Darío Santos, Franco Simini.

Dicha investigación tiene como objetivo analizar el efecto inmediato de un estímulo externo en las variables espacio temporales de la marcha de individuos con enfermedad de Parkinson.
Procedimiento: El proyecto consta de una fase de recolección de información de la marcha de pacientes con Parkinson, y posteriormente una fase de prueba que se llevarán a cabo en el Laboratorio de Marcha del Hospital de Clínicas. Se utilizará un dispositivo IMU (Inertial Measurement Unit) colocado en el tobillo de los pacientes, desarrollado por el equipo interdisciplinario en el Núcleo de Ingeniería Biomédica (NIB) de la facultad de Medicina e Ingeniería y la cátedra de rehabilitación y medicina física. Este dispositivo, recibe datos de aceleración, velocidad angular y magnetómetro de cada IMU, los procesa en tiempo real determinando las fases de la marcha y sus parámetros espacio-temporales (longitud del paso, cadencia, velocidad). De esta forma, PARKIBIP emite un estímulo mecánico periódico con una latencia ajustable al tiempo del paso basal (marcha habitual) a la que Ud. camina.

Se hará un registro de su marcha en tres ocasiones dentro de una misma sesión:
\begin{itemize}
    \item Caminando a una velocidad auto-seleccionada (como camina habitualmente)
    \item Caminando a una velocidad auto-seleccionada con el sensor inactivo 
    \item Caminando a una velocidad auto-seleccionada, con PARKIBIP activo y un estímulo mecánico programado con la velocidad media de la marcha auto-seleccionada.
\end{itemize}

Además de utilizar PARKIBIP, se le colocarán marcadores (esferas de 2 cm de diámetro) que serán pegadas con cinta adhesiva sobre la piel en la proximidad de las articulaciones de sus piernas. Durante su marcha las esferas serán filmadas mediante cámaras infrarrojas, que solamente captan los marcadores y no su cuerpo, por lo que se preserva el anonimato de su identidad. Este procedimiento es inocuo y no presenta riesgo para su salud.

Tenga presente que su participación es voluntaria. Usted no está obligado a formar parte de este proyecto, y en caso que opte por no participar su decisión será respetada. Asimismo, aún cuando usted decida involucrarse, podrá en cualquier momento y sin ninguna consecuencia desistir.

Los datos que se recaben son confidenciales, por lo que únicamente los conocerá el grupo de investigadores.

Su participación es gratuita, si usted decide participar no recibirá ningún beneficio económico.
