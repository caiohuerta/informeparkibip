\chapter{Escala Unificada de Clasificación de la Enfermedad de Parkinson (MDS-UPDRS) \cite{MDS-UPDRS2008}}\label{appendix:MDS-UPDRS}

\section{Estado Mental, de comportamiento y estado de ánimo}

\begin{table}[H]
\begin{center}
\begin{tabular}{|p{1cm}|p{11cm}|}
\hline
0 & Ninguna \\\hline
1 & Leve, olvido con recuerdo parcial de algunos hechos sin otras dificultades \\\hline
2 & Pérdida moderada de memoria, con desorientación y moderada dificultad en el tratamiento de problemas complejos \\\hline
3 & Pérdida grave de memoria con desorientación temporal y a menudo espacial, grave dificultad con los problemas \\\hline
4 & Pérdida grave de memoria con orientación sólo para personas, incapaz de juzgar o resolver problemas\\\hline
\end{tabular}
\caption{Deterioro intelectual}
\end{center}
\end{table}

\begin{table}[H]
\begin{center}
\begin{tabular}{|p{1cm}|p{11cm}|}
\hline
0 & Ninguno \\\hline
1 & Sueño Intenso\\\hline
2 & Alucinaciones «benignas» con retención de las mismas \\\hline
3 & Alucinaciones más frecuentes sin retención, pueden interferir con la actividad Diaria \\\hline
4 & Alucinaciones persistentes, ilusiones o psicosis floridas\\\hline
\end{tabular}
\caption{Trastornos del pensamiento}
\end{center}
\end{table}

\begin{table}[H]
\begin{center}
\begin{tabular}{|p{1cm}|p{11cm}|}
\hline
0 & No presente \\\hline
1 & Periodos de tristeza o culpabilidad superiores a lo normal, nunca presentes durante más de unos días o una semana \\\hline
2 & Depresión persistente durante más de una semana \\\hline
3 & Síntomas vegetativos (insomnio, anorexia, abulia, pérdida de peso) \\\hline
4 & Síntomas vegetativos con tendencias suicidas\\\hline
\end{tabular}
\caption{Depresión}
\end{center}
\end{table}

\begin{table}[H]
\begin{center}
\begin{tabular}{|p{1cm}|p{11cm}|}
\hline
0 & Normal \\\hline
1 & Menos afirmativo, más pasivo \\\hline
2 & Pérdida de iniciativa o desinterés en actitudes electivas \\\hline
3 & Pérdida de iniciativa o desinterés en la rutina diaria \\\hline
4 & Abandono, pérdida completa de motivación\\\hline
\end{tabular}
\caption{Motivación, iniciativa}
\end{center}
\end{table}

\section{Actividades de la vida diaria} 

\begin{table}[H]
\begin{center}
\begin{tabular}{|p{1cm}|p{11cm}|}
\hline
0 & Normal \\\hline
1 & Levemente afectado1, sin dificultad para ser entendido \\\hline
2 & Moderadamente afectado, ocasionalmente debe pedírsele que repita las cosas \\\hline
3 & Gravemente afectado, se le pide frecuentemente que repita las cosas \\\hline
4 & Ininteligible la mayor parte del tiempo\\\hline
\end{tabular}
\caption{Lenguaje}
\end{center}
\end{table}

\begin{table}[H]
\begin{center}
\begin{tabular}{|p{1cm}|p{11cm}|}
\hline
0 & Normal \\\hline
1 & Leve \\\hline
2 & Moderada excesiva salivación, babeo nocturno \\\hline
3 & Marcado exceso de salivación con algo de babe \\\hline
4 & Marcado babeo, requiere constante de pañuelo \\\hline
\end{tabular}
\caption{Salivación}
\end{center}
\end{table}

\begin{table}[H]
\begin{center}
\begin{tabular}{|p{1cm}|p{11cm}|}
\hline
0 & Normal \\\hline
1 & Levemente pequeña o lenta \\\hline
2 & Todas las palabras pequeñas pero legibles \\\hline
3 & Gravemente afectada, no son legibles todas las palabras \\\hline
4 & Mayoritariamente ilegibles\\\hline
\end{tabular}
\caption{Escritura}
\end{center}
\end{table}

\begin{table}[H]
\begin{center}
\begin{tabular}{|p{1cm}|p{11cm}|}
\hline
0 & Normal \\\hline
1 & Lento y poco hábil, pero se vale solo \\\hline
2 & Puede cortar la mayoría de alimentos, para algunos necesita ayuda\\\hline
3 & Le deben cortar la comida, pero puede alimentarse solo \\\hline
4 & Necesita que lo alimenten\\\hline
\end{tabular}
\caption{Cortar alimentos/manejar utensilios}
\end{center}
\end{table}

\begin{table}[H]
\begin{center}
\begin{tabular}{|p{1cm}|p{11cm}|}
\hline
0 & Normal \\\hline
1 & Lento, pero sin ayuda \\\hline
2 & Ocasionalmente necesita ayuda\\\hline
3 & Necesita considerable ayuda, aunque puede hacer algunas cosas solo \\\hline
4 & Necesita ayuda completa\\\hline
\end{tabular}
\caption{Vestir}
\end{center}
\end{table}

\begin{table}[H]
\begin{center}
\begin{tabular}{|p{1cm}|p{11cm}|}
\hline
0 & Normal \\\hline
1 & Lento, pero sin ayuda \\\hline
2 & Necesita ayuda con la ducha o el baño, o es muy lento en el cuidado de la higiene\\\hline
3 & Necesita ayuda para lavarse, cepillarse los dientes, ir al baño\\\hline
4 & Necesita ayuda completa\\\hline
\end{tabular}
\caption{Higiene}
\end{center}
\end{table}

\begin{table}[H]
\begin{center}
\begin{tabular}{|p{1cm}|p{11cm}|}
\hline
0 & Normal \\\hline
1 & Lento pero sin ayuda \\\hline
2 & Puede volverse o ajustar las sábanas pero con gran dificultad\\\hline
3 & No puede volverse o ajustarse las sábanas solo \\\hline
4 & Necesita ayuda completa\\\hline
\end{tabular}
\caption{ Volverse en la cama/ajustar las sábanas}
\end{center}
\end{table}

\begin{table}[H]
\begin{center}
\begin{tabular}{|p{1cm}|p{11cm}|}
\hline
0 & Ninguna \\\hline
1 & Rara \\\hline
2 & Ocasionales, menos de una por día \\\hline
3 &  Cae en promedio una vez por día \\\hline
4 & Más de una por día\\\hline
\end{tabular}
\caption{Caídas sin relación con el freezing}
\end{center}
\end{table}

\begin{table}[H]
\begin{center}
\begin{tabular}{|p{1cm}|p{11cm}|}
\hline
0 & Normal \\\hline
1 & Raro, puede haber duda \\\hline
2 & Caídas ocasionales por freezing\\\hline
3 & Frecuente freezing, caídas ocasionales \\\hline
4 & Frecuentes caídas por freezing\\\hline
\end{tabular}
\caption{Freezing al caminar}
\end{center}
\end{table} 

\begin{table}[H]
\begin{center}
\begin{tabular}{|p{1cm}|p{11cm}|}
\hline
0 & Normal \\\hline
1 & Leve dificultad, arrastra las piernas o disminuye el balanceo de los brazos \\\hline
2 & Dificultad moderada, sin requerir ayuda\\\hline
3 & Afectación grave, que requiere asistencia \\\hline
4 & No puede andar, incluso con ayuda\\\hline
\end{tabular}
\caption{Andar}
\end{center}
\end{table}

\begin{table}[H]
\begin{center}
\begin{tabular}{|p{1cm}|p{11cm}|}
\hline
0 & Ausente \\\hline
1 & Leve e infrecuente, no molesta al paciente \\\hline
2 & Moderado, molesto para el paciente\\\hline
3 & Grave, interfiere con muchas actividades \\\hline
4 & Marcado, interfiere con muchas actividades\\\hline
\end{tabular}
\caption{Temblor}
\end{center}
\end{table}

\begin{table}[H]
\begin{center}
\begin{tabular}{|p{1cm}|p{11cm}|}
\hline
0 & Ninguna \\\hline
1 & Ocasionalmente tiene insensibilidad, hormigueo y leve dolor \\\hline
2 & Frecuente pero no estresante\\\hline
3 & Sensación de dolor frecuente \\\hline
4 & Dolor insoportable\\\hline
\end{tabular}
\caption{Molestias sensoriales relacionadas con el parkinsonismo}
\end{center}
\end{table}

\section{Examen motor} 

\begin{table}[H]
\begin{center}
\begin{tabular}{|p{1cm}|p{11cm}|}
\hline
0 & Normal \\\hline
1 & Leve pérdida de expresión, dicción, volumen \\\hline
2 & Monótono, mal articulado pero comprensible\\\hline
3 & Marcada dificultad, difícil de entender \\\hline
4 & Ininteligible\\\hline
\end{tabular}
\caption{Lenguaje}
\end{center}
\end{table}

\begin{table}[H]
\begin{center}
\begin{tabular}{|p{1cm}|p{11cm}|}
\hline
0 & Normal \\\hline
1 & Leve hipomimia \\\hline
2 & Leve pero definida disminución anormal de la expresión\\\hline
3 & Moderada hipomimia, labios separados parte del tiempo \\\hline
4 & Cara fija, labios separados 1/2 cm o más, con pérdida completa de expresión \\\hline
\end{tabular}
\caption{Expresión facial}
\end{center}
\end{table}

\begin{table}[H]
\begin{center}
\begin{tabular}{|p{1cm}|p{11cm}|}
\hline
0 & Ausente \\\hline
1 & Leve e infrecuente \\\hline
2 & Necesita ayuda con la ducha o el baño, o es muy lento en el cuidado de la higiene\\\hline
3 & Necesita ayuda para lavarse, cepillarse los dientes, ir al baño \\\hline
4 & Necesita ayuda completa \\\hline
\end{tabular}
\caption{Temblor en reposo}
\end{center}
\end{table}

\begin{table}[H]
\begin{center}
\begin{tabular}{|p{1cm}|p{11cm}|}
\hline
0 & Ausente \\\hline
1 & Leve, presente con acción \\\hline
2 & Moderado, presente con acción\\\hline
3 & Moderado, presente con acción y manteniendo la postura \\\hline
4 & Marcado, interfiere con la alimentación\\\hline
\end{tabular}
\caption{Temblor postural o de acción en manos}
\end{center}
\end{table}

\begin{table}[H]
\begin{center}
\begin{tabular}{|p{1cm}|p{11cm}|}
\hline
0 & Ausente \\\hline
1 & Leve o sólo con actividad \\\hline
2 & Leve/moderada\\\hline
3 &  Marcada, en todo el rango de movimiento \\\hline
4 & Grave\\\hline
\end{tabular}
\caption{Rigidez}
\end{center}
\end{table}


\begin{table}[H]
\begin{center}
\begin{tabular}{|p{1cm}|p{11cm}|}
\hline
0 & Normal \\\hline
1 & Leve lentitud y/o reducción en amplitud \\\hline
2 & Dificultad moderada\\\hline
3 & Dificultad grave \\\hline
4 & Apenas puede realizarlos\\\hline
\end{tabular}
\caption{Tocarse la punta de los dedos (Finger tapping, el paciente golpea el pulgar con el índice en rápida sucesión y con la mayor amplitud posible; realizar con cada mano por separado)}
\end{center}
\end{table}

\begin{table}[H]
\begin{center}
\begin{tabular}{|p{1cm}|p{11cm}|}
\hline
0 & Normal \\\hline
1 & Leve lentitud y/o reducción en amplitud \\\hline
2 & Dificultad moderada\\\hline
3 & Dificultad grave \\\hline
4 & Apenas puede realizarlos\\\hline
\end{tabular}
\caption{Movimientos de la mano (abrir y cerrar las manos en rápida sucesión)}
\end{center}
\end{table}

\begin{table}[H]
\begin{center}
\begin{tabular}{|p{1cm}|p{11cm}|}
\hline
0 & Normal \\\hline
1 & Leve lentitud y/o reducción en amplitud \\\hline
2 & Dificultad moderada\\\hline
3 & Dificultad grave \\\hline
4 & Apenas puede realizarlos\\\hline
\end{tabular}
\caption{Agilidad en la pierna (movimientos con el talón sobre el suelo, la amplitud debería ser de 3 pulgadas)}
\end{center}
\end{table}

\begin{table}[H]
\begin{center}
\begin{tabular}{|p{1cm}|p{11cm}|}
\hline
0 & Normal \\\hline
1 & Lento, puede necesitar más de un intento \\\hline
2 & Se empuja hacia arriba con los brazos o la silla\\\hline
3 & Tiende a caer hacia atrás, puede necesitar muchos intentos pero puede levantarse sin ayuda \\\hline
4 & Incapaz de levantarse sin ayuda\\\hline
\end{tabular}
\caption{Levantarse de una silla (con brazos cruzados)}
\end{center}
\end{table}


\begin{table}[H]
\begin{center}
\begin{tabular}{|p{1cm}|p{11cm}|}
\hline
0 & Normal erecto \\\hline
1 & Levemente inclinado, podría ser normal para una persona mayor \\\hline
2 & Anormal: inclinado, puede que hacia algún lado\\\hline
3 & Grave inclinación con escoliosis \\\hline
4 & Marcada flexión con postura muy anormal\\\hline
\end{tabular}
\caption{Postura}
\end{center}
\end{table}


\begin{table}[H]
\begin{center}
\begin{tabular}{|p{1cm}|p{11cm}|}
\hline
0 & Normal \\\hline
1 & Anda lentamente \\\hline
2 & Anda con dificultad, con poca o sin ayuda, algún balanceo, pasos cortos o propulsión \\\hline
3 & Afectación grave, necesita ayuda frecuente \\\hline
4 & No puede andar\\\hline
\end{tabular}
\caption{Marcha}
\end{center}
\end{table}

\begin{table}[H]
\begin{center}
\begin{tabular}{|p{1cm}|p{11cm}|}
\hline
0 & Normal \\\hline
1 & Se recupera sin ayuda \\\hline
2 & Caería si no se sostiene\\\hline
3 & Se cae espontáneamente \\\hline
4 & Imposible mantenerse de pie\\\hline
\end{tabular}
\caption{Estabilidad postural (test de retropulsión)}
\end{center}
\end{table}

\begin{table}[H]
\begin{center}
\begin{tabular}{|p{1cm}|p{11cm}|}
\hline
0 & Nada \\\hline
1 & Mínima lentitud, podría ser normal \\\hline
2 & Leve lentitud y escasez de movimientos, definitivamente anormales, o disminuye la amplitud de movimientos\\\hline
3 & Moderada lentitud, escasez de movimientos, o disminuye la amplitud\\\hline
4 & Marcada lentitud, escasez de movimientos, o disminuye la amplitud de movimientos\\\hline
\end{tabular}
\caption{Bradicinesia/hipocinesia}
\end{center}
\end{table}

\section{Complicaciones del tratamiento (en la semana anterior)}
Discinesias.
\begin{table}[H]
\begin{center}
\begin{tabular}{|p{1cm}|p{11cm}|}
\hline
0 & No hay \\\hline
1 & 1\% a 25\% del día \\\hline
2 & 26\% a 50\% del día\\\hline
3 & 5l\% a 75\% del día \\\hline
4 & 76\% a 100\% del día\\\hline
\end{tabular}
\caption{Duración: ¿En qué proporción de las horas de vigilia del día están presentes las discinesias? (información por anamnesis)}
\end{center}
\end{table}

\begin{table}[H]
\begin{center}
\begin{tabular}{|p{1cm}|p{11cm}|}
\hline
0 & No incapacitantes \\\hline
1 & Discretamente incapacitantes \\\hline
2 & Moderadamente incapacitantes\\\hline
3 & Muy incapacitantes \\\hline
4 & Totalmente invalidantes\\\hline
\end{tabular}
\caption{Incapacidad: ¿Qué grado de incapacidad producen las discinesias? (información por anamnesis; puede modificarse por la exploración)}
\end{center}
\end{table}

\begin{table}[H]
\begin{center}
\begin{tabular}{|p{1cm}|p{11cm}|}
\hline
0 & No hay discinesias dolorosas \\\hline
1 & Leve \\\hline
2 & Moderado\\\hline
3 & Intenso \\\hline
4 & Marcado\\\hline
\end{tabular}
\caption{Discinesias dolorosas: ¿Cuánto dolor producen las discinesias?}
\end{center}
\end{table}

\begin{table}[H]
\begin{center}
\begin{tabular}{|p{1cm}|p{11cm}|}
\hline
0 & Si \\\hline
1 & No \\\hline
\end{tabular}
\caption{Presencia de distonía en las primeras horas de la mañana (información por anamnesis)}
\end{center}
\end{table}

Fluctuaciones clínicas.
\begin{table}[H]
\begin{center}
\begin{tabular}{|p{1cm}|p{11cm}|}
\hline
0 & Si \\\hline
1 & No \\\hline
\end{tabular}
\caption{ ¿Hay algunos periodos off predecibles que se instauran en un momento determinado tras una dosis de medicación?}
\end{center}
\end{table}
\begin{table}[H]
\begin{center}
\begin{tabular}{|p{1cm}|p{11cm}|}
\hline
0 & Si \\\hline
1 & No \\\hline
\end{tabular}
\caption{¿Hay algunos periodos off imprevisibles en relación con las tomas de medicación?}
\end{center}
\end{table}

\begin{table}[H]
\begin{center}
\begin{tabular}{|p{1cm}|p{11cm}|}
\hline
0 & Si \\\hline
1 & No \\\hline
\end{tabular}
\caption{¿Hay periodos off que instauran de forma súbita? (por ejemplo,en pocos segundos)}
\end{center}
\end{table}

\begin{table}[H]
\begin{center}
\begin{tabular}{|p{1cm}|p{11cm}|}
\hline
0 & Ninguna \\\hline
1 & 1\% al 25\% del día \\\hline
2 & 26\% al 50\% del día\\\hline
3 & 51\% al 75\% del día \\\hline
4 & 76\% al 100\% del día\\\hline
\end{tabular}
\caption{Como promedio, ¿en qué proporción de las horas de vigilia del día está el paciente en off?}
\end{center}
\end{table}

Otras complicaciones.
\begin{table}[H]
\begin{center}
\begin{tabular}{|p{1cm}|p{11cm}|}
\hline
0 & Si \\\hline
1 & No \\\hline
\end{tabular}
\caption{ ¿Tiene el paciente anorexia, náuseas o vómitos?}
\end{center}
\end{table}

\begin{table}[H]
\begin{center}
\begin{tabular}{|p{1cm}|p{11cm}|}
\hline
0 & Si \\\hline
1 & No \\\hline
\end{tabular}
\caption{¿Tiene el paciente algún trastorno del sueño? (por ejemplo, insomnio o hipersomnia)}
\end{center}
\end{table}

\begin{table}[H]
\begin{center}
\begin{tabular}{|p{1cm}|p{11cm}|}
\hline
0 & Si \\\hline
1 & No \\\hline
\end{tabular}
\caption{¿Tiene el paciente ortostatismo sintomático?}
\end{center}
\end{table}

\section{Escala Schwab \& England }

\begin{table}[H]
\begin{center}
\begin{tabular}{|p{1cm}|p{11cm}|}
\hline
\% & Es posible seleccionar números entre los porcentajes establecido \\\hline \hline
100 & Completamente independiente. Capaz de realizar cualquier tarea sin dificultad, lentitud o alteraciones \\\hline
 90 & Completamente independiente. Puede tardar el doble de lo normal en realizar una tarea \\\hline
 80 & Independiente en la mayoría de tareas. Tarda el doble. Consciente de su dificultad y enlentecimiento \\\hline
 70 & No completamente independiente. En algunas tareas tarda tres o cuatro veces más de lo normal \\\hline
 60 & Alguna dependencia. Puede hacer la mayoría de tareas, pero muy lentamente y con mucho esfuerzo \\\hline
 50 & Más dependiente. Necesita ayuda en la mitad de las tareas cotidianas. Dificultad para todo \\\hline
 40 & Muy dependiente. Sólo puede realizar algunas tareas sin ayuda. Con mucho esfuerzo puede realizar alguna tarea. Necesita mucha ayuda \\\hline
 30 & Ninguna tarea solo. Grave invalidez \\\hline
 20 & Totalmente dependiente. Puede ayudar algo en algunas actividades\\\hline 
 10 & Dependiente. Inválido \\\hline
 0 & Las funciones vegetativas como tragar, vejiga e intestino no funcionan, postrado en cama \\\hline
\end{tabular}
\caption{Escala de AVD}
\end{center}
\end{table}

\section{Estadificación Modificada Hoehn \& Yahr}

\begin{table}[H]
\begin{center}
\begin{tabular}{|p{2cm}|p{11cm}|}
\hline
Estadio 0 & No hay signos de enfermedad \\\hline
Estadio 1 & Enfermedad unilateral \\\hline
Estadio 1.5 & Afectación axial y unilateral \\\hline
Estadio 2 & Enfermedad Bilateral, sin deterioro del equilibrio\\\hline
Estadio 2.5 & Enfermedad bilateral leve, con recuperación en la prueba de tracción \\\hline
Estadio 3 & Enfermedad bilateral leve a moderada, cierta inestabilidad postural; independencia física \\\hline
Estadio 4 & Discapacidad grave, aún puede caminar o estar de pie sin asistencia\\\hline
Estadio 5 & Destinado a la silla de ruedas o postrado en cama, a no ser que sea asistido\\\hline
\end{tabular}
\caption{Escala de estadios}
\end{center}
\end{table}

